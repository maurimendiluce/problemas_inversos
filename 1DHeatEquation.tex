\documentclass{beamer}
\usepackage[spanish]{babel}
\usepackage[latin1]{inputenc}
\usepackage{amsmath,amssymb,amsfonts}
\usepackage{amsthm}
\usepackage{mathrsfs}
\usepackage{graphicx}
\usepackage{color}
\usepackage{hyperref} %para poner links y que sean cliqueables en el pdf

%\setlength{\textwidth}{11cm} \setlength{\textheight}{16cm}

%\topmargin-1cm \oddsidemargin-1cm \evensidemargin-1cm
%\textwidth17cm \textheight22.25cm

\usetheme{Singapore}

\spanishdecimal{.}

\begin{document}


\begin{frame}
\begin{center}
\large{\textbf{\color{red} 1D Inverse Heat Equation}}
\end{center}

An illustrative example is the inversion for the initial condition for a one-dimensional heat equation.

We consider a rod of lenght $L$,  and we are interested in reconstructing the initial temperature profile $m$ given some noisy measurements $d$ of the temperature profile at a later time $T$. \\
\end{frame}

\begin{frame}
{\color{red}\underline{ Forward problem}}

Given

\begin{itemize}
\item The initial temperature profile $u(x,0)=m(x)$,
\item The termal diffusivity $k$,
\item A prescribed temperature $u(0,t)=u(L,t)=0$ at the extremities of the rod,
\end{itemize}
solve the heat equation

\begin{eqnarray*}
\left\{
\begin{aligned}
\frac{\partial u}{\partial t} - k \frac{\partial ^2 u}{\partial x^2} = 0  \quad \forall x &\in (0.L), \,\, \forall t \in (0,T) \\
u(x,0) = m(x) \quad \forall x &\in [0,L] \\
u(0,t)=u(L,t) = 0 \quad \forall t &\in (0,T],
\end{aligned}
\right.
\end{eqnarray*}

and observe the temperature at the final time $T$: $\mathcal{F}(m)=u(x,T).$ \\
\end{frame}

\begin{frame}
{\color{red} \underline{Analytical solution to the forward problem}}

Verify that if $m(x)=\sin(n\frac{\pi}{L}x), \quad n=1,2,3, \cdots$

then $u(x,t)=e^{-k(n \frac{\pi}{L})^2t} \sin(n\frac{\pi}{L}x)$ is the unique solution to the heat equation.  \\

\vspace{0.5cm}

{\color{red} \underline{Inverse problem} }

Given the forward model $\mathcal{F}$ and a noisy measurament $d$ of the temperature profile at time $T$, find the initial temperature profile $m$ such that
$$\mathcal{F}(m)=d.$$
\end{frame}

\begin{frame}
{\color{red} \underline{Ill-posedness of the inverse problem}}

Consider a perturbation $\delta m(x)=\epsilon \sin(n\frac{\pi}{L}x)$, where $\epsilon>0$ and $n=1,2,3, \cdots$.

Then, by linearity of the forward model $\mathcal{F}$, the corresponding perturbation $\delta d(x)=\mathcal{F}(m+\delta m) - \mathcal{F}(m)$ is $\delta d(x)=\epsilon e^{-k(n\frac{\pi}{L})^2T} \sin(n\frac{\pi}{L}x)$, which converges to zero as $n \longrightarrow + \infty$.

Hence the ratio between $\delta m$ and $\delta d$ can become arbitrary large, which shows that the stability requirement for well-posedness can not be satisfied. \\
\end{frame}

\begin{frame}
{\color{red} \underline{Discretization}}

We use finite differences in space and Implicit Euler in time. \\

{\color{red} $\longrightarrow$ Semidiscretization in space }

We divide the $[0,L]$ interval in $n_x$ subintervals of the same lenght $h=\frac{L}{n_x}$, and we denote with $u_j(t):=u(j \cdot h,t)$ the value of the temperature at point $x_j=j \cdot h$ and time $t$.

We use a centered finite difference approximation of the second derivative in space and write
$\frac{\partial u_{j} (t)}{\partial t } -  k \frac{u_{j-1} (t) - 2 u_{j} (t)+ u_{j+1} (t)}{h^2}$ for $ j=1,2, \cdots , n_{x}-1$, with the boundary condition $u_0 (t)=u_{n_x}(t)=0$.
\end{frame}

\begin{frame}
By letting $\mathbf{u}(t)= \begin{pmatrix}
u_1(t) \\
u_2(t) \\
\vdots \\
u_{n_x-1}(t)\\
\end{pmatrix}$

be the vector collecting the values of the temperature $u$ at the points $x_j=j \cdot h$, we then write the system of ordinary differential equations (ODEs):
$$\frac{\partial}{\partial t}\mathbf{u}(t)+K \mathbf{u}(t)=\mathbf{0},$$

where $K \in \mathbb{R}^{(n_x - 1)\times(n_x - 1)}$ is the tridiagonal matrix given by

$$K=\frac{k}{h^2} \begin{pmatrix}
2 & -1 & & & &  \\
-1 & 2 & -1 & & & \\
 & -1 & 2 & -1 & & \\
 & & \cdots & \cdots & \cdots & \\
 & & & -1 & 2 & -1\\
 & & & & -1 & 2
\end{pmatrix}.$$
\end{frame}

\begin{frame}

{\color{red} $\longrightarrow$ Time discretization}

We subdivide the time interval (0,T] in $n_t$ time step of size $\Delta t=\frac{T}{n_t}$. By letting $\mathbf{u}^{(i)}=\mathbf{u}(i\cdot\Delta t)$ denote the discretized temperature profile at time $t_i=i \cdot \Delta t$, the Implicit Euler scheme reads $$\frac{\mathbf{u}^{(i+1)}-\mathbf{u}^{(i)}}{ \Delta t} + K\mathbf{u}^{(i+1)}=\mathbf{0}, \quad \mbox{ for } i=0,1,\cdots, n_t - 1.$$

After simple algebraic manipulations and exploiting the initial condition $u(x,0)=m(x)$, we then obtain 
\begin{eqnarray*}
\left\{
\begin{aligned}
\mathbf{u}^{(i+1)} &= (I+\Delta t \cdot K)^{-1 }\mathbf{u}^{(i)}  \\
\mathbf{u}^{(0)} &=\mathbf{ m} 
\end{aligned}
\right.
\end{eqnarray*}

or equivalently, $\mathbf{u}^{(i)}=(I+ \Delta t \cdot  K)^{-i}\mathbf{m}$.

$${\color{green} \framebox[2\width]{C\'odigo Parte 1}}$$

{\color{cyan} assembleMatrix} generates the finite difference matrix $(I+\Delta  t \cdot K)$ and {\color{cyan} solveFwd} evaluates the forward model $F\mathbf{m}= (I+\Delta t \cdot K)^{-n_t}$ $\mathbf{m}$.
\end{frame}

\begin{frame}
{\color{red} \underline{A naive solution to the inverse problem}}

If $\mathcal{F}$ is invertible a naive solution to the inverse problem $\mathcal{F}m=d$ is simply to set $m=\mathcal{F}^{-1}d$. The function {\color{cyan}naiveSolveInv} computes the solution of the discretized inverse problem $\mathbf{m}=F^{-1}\mathbf{d}$ as $\mathbf{m}=(I+\Delta t K)^{n_t} \mathbf{d}$. \\

It can be seen that:

\begin{itemize}
\item For a very coarse mesh $(n_x = 20)$ and no measurement noise $(noise\_std\_dev = 0.0)$ the naive solution is quite good.
\item For a finer mesh $(n_x = 100)$ and/or even small measurement noise $(noise\_std\_dev = 1e-4)$ the naive solution is garbage.
\end{itemize}
$${\color{green} \framebox[2\width]{C\'odigo Parte 2}}$$
\end{frame}

\begin{frame}
{\color{red} Why does the naive solution fail?}

Let $v_n=\sqrt{\frac{2}{L}}\sin(n\frac{\pi}{L}x)$ with $n=1,2,3,\cdots$, then we have that
$\mathcal{F}v_n=\lambda_n v_n,$ where the eigenvalues $\lambda_n=e^{-kT(\frac{\pi}{L}n)^2}$. \\

$\mathbf{Note \,\, 1}$:

\begin{itemize}
\item    Large eigenvalues $\lambda_n$ corresponds to smooth eigenfunctions $v_n$,
\item Small eigenvalues $\lambda_n$ corresponds to oscillatory eigenfuctions $v_n$.
\end{itemize}

The eigenvalues $\lambda_n$ decays extremely fast, that is the matrix $F$, discretization of the forward model $\mathcal{F}$, is extremely ill conditioned.
$${\color{green} \framebox[2\width]{C\'odigo Parte 3}}$$
\end{frame}

\begin{frame}
$\mathbf{Note \,\, 2}$:

 The functions $v_n, n=1,2,3, \cdots$, form an orthonormal basis of $L^{2}([0,1])$.

That is, every function $f \in L^{2}([0,1])$ can be written as

$$f =\sum_{n=1}^{\infty} \alpha_n \,\, v_n, \mbox{ where } \alpha_n=\int_0^1 \,\, f \,\, v_n\,\, dx.$$
Consider now the noisy problem

$$d = \mathcal{F} m_{\mbox{true}} + \eta,$$
where

\begin{itemize}
\item $d$ is the data (noisy measurements)
\item $\eta$ is the noise: $\eta(x) = \sum_{n=1}^{ \infty} \eta_n \,\, v_n(x)$
\item $m_{\mbox{true}}$ is the true value of the parameter that generated the data
\item $\mathcal{F}$ is the forward heat equation
\end{itemize}
\end{frame}

\begin{frame}
Then, the naive solution to the inverse problem $\mathcal{F}m=d$  is

$$m=\mathcal{F}^{-1}d = \mathcal{F}^{-1} (\mathcal{F} m_{\mbox{true}}+\eta)=m_{\mbox{true}}+ \mathcal{F}^{-1} \eta =$$ $$m_{\mbox{true}}+ \mathcal{F}^{-1} \sum_{n=1}^{ \infty} \eta_n \,\, v_n = m_{\mbox{true}}+ \sum_{n=1}^{ \infty} \frac{\eta_n}{\lambda_n} \,\, v_n.$$

If the coefficients $\eta_n = \int_{0}^{1} \eta(x) \,\, v_n(x) \,\,dx$  do not decay sufficiently fast with respect to the eigenvalues $\lambda_n$, then the naive solution is unstable.

This implies that oscillatory components can not reliably be reconstructed from noisy data since they correspond to small eigenvalues. \\
\end{frame}

\begin{frame}
{\color{red} \underline{Regularization by filtering}}

Remedy is to dampen the terms corresponding to small eigenvalues and write

$$m \approx \sum_{n=1}^{\infty} \omega(\lambda_n^2) \,\, \lambda_n^{-1} \,\, \delta_n \,\, v_n,$$

where

\begin{itemize}
\item $\delta_n=\int_0^1 d(x)\,\, v_n(x) \,\, dx$  denotes the coefficients of the data $d$  in the basis $ \{ v_n \}_{n=1}^{\infty}$,
\item $\omega(\lambda_n^2)$ is a filter function that allows to drop/stabilize the terms corresponding to small $\lambda_n$
\end{itemize}
\end{frame}

\begin{frame}
Consider $\alpha$ is a regularization parameter,
\begin{itemize}
\item {\color{red} Truncated Singular Value Decomposition:} \begin{eqnarray*}
\omega_{\alpha}(\lambda^2) = \left\{
\begin{aligned}
&1, \quad  \mbox{ if } \lambda^2 \geq \alpha \\
&0, \quad \mbox{ otherwise }
\end{aligned},
\right.
\end{eqnarray*} Then, we have $m_{\mbox{svd}}=\sum_{n=1}^{N} \lambda_n^{-1} \,\, \delta_n \,\, v_n$, where $N$ is largest index such that $\lambda_n^2 \geq \alpha$ (assuming that $\lambda_n$ are sorted in a decreasing order). 
\item {\color{red} Tikhonov filter:} $\omega_{\alpha}(\lambda^2)=\frac{\lambda^2}{\lambda^2 + \alpha}$, ($\omega_{\alpha}(\lambda^2)$ is close to $1$ when $\lambda \gg \alpha$, close to $0$ when $\lambda \ll \alpha)$.
Then we have $m_{\mbox{tikh}} = \sum_{n=1}^{\infty} \frac{\lambda_n}{\lambda_n^2 + \alpha} \,\, \delta_n \,\, v_n$.
\end{itemize}

$${\color{green} \framebox[2\width]{C\'odigo Parte 4}}$$
\end{frame}

\end{document}

